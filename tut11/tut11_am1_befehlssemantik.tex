\documentclass[ngerman,a4paper, 12pt]{article}
\usepackage[top=2.5cm,bottom=2.5cm,left=2.5cm,right=2.5cm]{geometry}

\usepackage[ngerman]{babel}
\usepackage[utf8]{inputenc}

\usepackage{chngcntr}

\usepackage{fancyhdr} 	% customize header / footer
\usepackage{titlesec} 	% customize titles
\usepackage{tabularx} 	% tabularx-environment (explicitly set width of columns)
\usepackage{tabu}
\usepackage{longtable} 	% Tabellen mit Seitenumbrüchen
\usepackage{multirow}
\usepackage{booktabs}	% improved rules
\usepackage{colortbl} 

\usepackage[scale=1]{opensans}
\newcommand*{\osfamily}{\fontfamily{fos}\selectfont}
\DeclareTextFontCommand{\textos}{\osfamily}

%%%%%%%%%%%%%%%%%%%%%%%%%%%%%%%%%%%%%%%%%%%%%%%%%
% math related packages
% basic ams-math and enhancments
\usepackage{amsmath,amssymb,amsfonts,mathtools}
\usepackage{blkarray}
% add some font-related stuff
\usepackage{lmodern}
\usepackage{latexsym}
\usepackage{marvosym} 	% lightning (contradiction)
\usepackage{stmaryrd} 	% Lightning symbol
\usepackage{bbm} 		% unitary matrix
\usepackage{wasysym}	% add some symbols

% further support for different equation setting
\usepackage{xfrac}		% sfrac -> fractions e.g. 3/4


\usepackage[amsthm,thmmarks,hyperref]{ntheorem}
\usepackage[ntheorem,framemethod=TikZ]{mdframed}


%%%%%%%%%%%%%%%%%%%%%%%%%%%%%%%%%%%%%%%%%%%%%%%%%
% graphics-related packages
\usepackage[table,dvipsnames]{tudscrcolor} 



%%%%%%%%%%%%%%%%%%%%%%%%%%%%%%%%%%%%%%%%%%%%%%%%%
% text-related packages
% increase line spacing
\usepackage[onehalfspacing]{setspace} % increase row-space
\usepackage{ulem} 		% better underlines

% enumeration
\usepackage{enumerate}
\usepackage[inline]{enumitem} 		%customize label

% source code
\usepackage{listings}

%%%%%%%%%%%%%%%%%%%%%%%%%%%%%%%%%%%%%%%%%%%%%%%%%%%%%%%%%%%%%%%%%%%%%%%


%%%%%%%%%%%%%%%%%%%%%%%%%%%%%%%%%%%%%%%%%%%%%%%%%%%%%%%%%%%%%%%%%%%%%%%
%                          HYPERLINKS                                 %
%%%%%%%%%%%%%%%%%%%%%%%%%%%%%%%%%%%%%%%%%%%%%%%%%%%%%%%%%%%%%%%%%%%%%%%
\usepackage{bookmark}		% pdf-bookmarks

%%%%%%%%%%%%%%%%%%%%%%%%%%%%%%%%%%%%%%%%%%%%%%%%%%%%%%%%%%%%%%%%%%%%%%%
\titleformat{\section}{\normalfont\sffamily\Large\bfseries}{\thesection}{1em}{}
\newcommand*\head{\rowfont{\bfseries}}

\newcommand{\menge}[1]{\left\{ #1 \right\}}

\begin{document}
	\pagestyle{empty}
	
	\section*{Befehlssematik der abstrakten Maschine AM${}_{\text{1}}$}
	
	\textbf{Adressberechnung} \\
	Sei $b \in \menge{\text{global}, \text{lokal}}$ und $r$ der aktuelle REF, dann berechnet sich die Adresse zu
	\begin{equation*}
	adr(r,b,o) = \begin{cases}
	r+o & \text{wenn } b = \text{lokal} \\
	o & \text{wenn } b = \text{global}
	\end{cases}
	\end{equation*}
	
	\renewcommand*{\arraystretch}{1.6}
	\noindent
	\begin{tabu}{p{2.5cm} | p{\dimexpr\linewidth-2.5cm}}
		\toprule
		\head Befehl & Auswirkungen  \\ 
		\midrule \midrule
		LOAD($b$,$o$) & Lädt den Inhalt von Adresse $adr(r,b,o)$ auf den Datenkeller, inkrementiere Befehlszähler \\
		STORE($b$,$o$) & Speichere oberstes Datenkellerelement an $adr(r,b,o)$, inkrementiere Befehlszähler \\
		WRITE($b$,$o$) & Schreibe Inhalt an Adresse $adr(r,b,o)$ auf das
		Ausgabeband, inkrementiere Befehlszähler \\
		READ($b$,$o$) & Lies oberstes Element vom Eingabeband, speichere
		an Adresse $adr(r,b,o)$, inkrementiere Befehlszähler \\
		LOADI($o$) & Ermittle Wert ($=b$) an Adresse $r+o$, Lade Inhalt
		von Adresse $b$ auf Datenkeller, inkrementiere Befehlszähler \\
		STOREI($o$) & Ermittle Wert ($=b$) an Adresse $r+o$, nimm
		oberstes Datenkellerelement, speichere dieses an
		Adresse $b$, inkrementiere Befehlszähler \\
		WRITEI($o$) & Ermittle Wert ($=b$) an Adresse $r+o$, schreibe den
		Inhalt an Adresse $b$ auf Ausgabeband, inkrementiere Befehlszähler \\
		READI($o$) & Ermittle Wert ($=b$) an Adresse $r+o$, lies das oberste
		Element vom Eingabeband, speichere es an Adresse $b$ , inkrementiere Befehlszähler \\
		LOADA($b$,$o$) & Lege $adr(r,b,o)$ auf Datenkeller, inkrementiere Befehlszähler \\			
		PUSH & oberstes Element vom Datenkeller auf Laufzeitkeller, Befehlszähler inkrementieren \\
		CALL $adr$ & Befehlszählerwert inkrementieren und auf LZK legen, Befehlszähler auf $adr$ setzen, REF auf LZK legen, REF auf Länge des LZK ändern \\
		INIT $n$ & $n$-mal $0$ auf den Laufzeitkeller legen \\
		RET $n$ & im LZK alles nach REF-Zeiger löschen, oberstes Element des LZK als REF setzen, oberstes Element des LZK als Befehlszähler setzen, $n$ Elemente von LZK löschen \\
		\bottomrule
	\end{tabu}
\end{document}